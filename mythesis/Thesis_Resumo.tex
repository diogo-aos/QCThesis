%%%%%%%%%%%%%%%%%%%%%%%%%%%%%%%%%%%%%%%%%%%%%%%%%%%%%%%%%%%%%%%%%%%%%%%%
%                                                                      %
%     File: Thesis_Resumo.tex                                          %
%     Tex Master: Thesis.tex                                           %
%                                                                      %
%     Author: Andre C. Marta                                           %
%     Last modified : 21 Jan 2011                                      %
%                                                                      %
%%%%%%%%%%%%%%%%%%%%%%%%%%%%%%%%%%%%%%%%%%%%%%%%%%%%%%%%%%%%%%%%%%%%%%%%

\section*{Resumo}

% Add entry in the table of contents as section
\addcontentsline{toc}{section}{Resumo}

Avanços na tecnologia permitem a recolha e armazenamento de quantidades e variedades de dados sem precedente.
A maior parte destes dados são armazenados eletrónicamente e existe interesse em realizar análise automática dos mesmos.
As técnicas de \emph{clustering} estão entre as mais populares para essa tarefa porque não assumem nada sobre a estrutura dos dados \emph{a priori}.
Dezenas de técnicas existem, mas, típicamente, uma só técnica não tem um bom desempenho em todos os conjuntos de dados devido às especifidades de cada um.
Técnicas de \emph{ensemble clustering} tentam responder a esse desafio ao combinar outros algoritmos.
Esta dissertação foca-se numa em particular, o \emph{Evidence Accumulation Clustering} (EAC).
O EAC é uma algorithm robusto que tem demonstrado bons desempenhos na literatura numa variedade de conjuntos de dados.
No entanto, esta robustez vem com um maior custo computacional associado.
A sua aplicação não só é mais lenta como está restrita a conjuntos de dados pequenos.
Assim, o objetivo desta dissertação é escalar o EAC, possibilitando a sua a aplicação a conjuntos de dados grandes (mais que $500 \: 000$ amostras), com tecnologia disponivel numa típica estação de trabalho.
Com isto em mente, várias abordagens foram exploradas: acelerar processamento com outros algoritmos (\emph{quantum clustering}), através de processamento paralelo (com GPU), escalar para maiores conjuntos de dados com algoritmos de memória externa (disco rígido) e explorando a natureza esparsa do EAC.
Uma contribuição de relevo é um método novo para construir uma matriz de esparsa específico ao EAC.
Os resultados mostram que a solução proposta é, efetivamente, aplicável a conjuntos de dados grandes e é significativamente mais rápida que a original para conjuntos pequenos.

\textbf{\Large Palavras-chave:} clustering, EAC, quantum clustering, K-Means, Single-Link, matriz esparsa, GPGPU





\cleardoublepage

