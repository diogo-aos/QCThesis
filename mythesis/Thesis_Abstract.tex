%%%%%%%%%%%%%%%%%%%%%%%%%%%%%%%%%%%%%%%%%%%%%%%%%%%%%%%%%%%%%%%%%%%%%%%%
%                                                                      %
%     File: Thesis_Abstract.tex                                        %
%     Tex Master: Thesis.tex                                           %
%                                                                      %
%     Author: Andre C. Marta                                           %
%     Last modified : 21 Jan 2011                                      %
%                                                                      %
%%%%%%%%%%%%%%%%%%%%%%%%%%%%%%%%%%%%%%%%%%%%%%%%%%%%%%%%%%%%%%%%%%%%%%%%

\section*{Abstract}

% Add entry in the table of contents as section
\addcontentsline{toc}{section}{Abstract}

Advances in technology allow for the collection and storage of unprecedented amount and variety of data.
Most of this data is stored electronically and there is an interest in automated analysis for generation of knowledge and new insights.
Since, often, the structure of the data is unknown, clustering techniques become particularly interesting for knowledge discovery and data mining, since they makes as few assumptions on the data as possible.
A vast body of work on these algorithms exist, yet, typically, no single algorithm is able to respond to the specificities of all data.
Ensemble clustering algorithm address this problem by combining other algorithms.
Evidence Accumulation Clustering (EAC) is a robust ensemble algorithm that has shown good results and is the focus of this dissertation.
However, this robustness comes with higher computational cost.
Its application is slower and restricted to smaller data sets.
Thus, the objective of this dissertation is to scale EAC, allowing its applicability to big data sets (more than $500 \: 000$ patterns), with technology available at a typical workstation.
Accordingly, several approaches were explored: speed-up with other algorithms (\emph{quantum clustering}) or parallel computing (with GPU) and reducing space complexity by using external memory (hard drive) algorithms and exploiting the sparse nature of EAC.
A relevant contribution is a novel method to build a sparse matrix specialized in EAC.
Results show that the proposed solution is applicable to large data sets and is significantly faster than the original for smaller data sets.

\vfill

\textbf{\Large Key words:} clustering, EAC, quantum clustering, K-Means, Single-Link, sparse matrix, GPGPU

\cleardoublepage

