%%%%%%%%%%%%%%%%%%%%%%%%%%%%%%%%%%%%%%%%%%%%%%%%%%%%%%%%%%%%%%%%%%%%%%%%
%                                                                      %
%     File: Thesis_Glossary.tex                                        %
%     Tex Master: Thesis.tex                                           %
%                                                                      %
%     Author: Andre C. Marta                                           %
%     Last modified : 30 Oct 2012                                      %
%                                                                      %
%%%%%%%%%%%%%%%%%%%%%%%%%%%%%%%%%%%%%%%%%%%%%%%%%%%%%%%%%%%%%%%%%%%%%%%%
%
% The definitions can be placed anywhere in the document body
% and their order is sorted by <symbol> automatically when
% calling makeindex in the makefile
%
% The \glossary command has the following syntax:
%
% \glossary{entry}
%
% The \nomenclature command has the following syntax:
%
% \nomenclature[<prefix>]{<symbol>}{<description>}
%
% where <prefix> is used for fine tuning the sort order,
% <symbol> is the symbol to be described, and <description> is
% the actual description.

% ----------------------------------------------------------------------

\glossary{name={\textbf{Qubit}},description={Quantum bit.}}
\glossary{name={\textbf{QK-Means}},description={Quantum K-Means.}}

\glossary{name={\textbf{PCA}},description={Principal Component Analysis.}}
\glossary{name={\textbf{PC}},description={Principal Component.}}
\glossary{name={\textbf{SVD}},description={Singular Value Decomposition.}}

\glossary{name={\textbf{EAC}},description={Evidence Accumulation Clustering.}}
\glossary{name={\textbf{HAC}},description={Hierarchical Agglomeration Clustering.}}
\glossary{name={\textbf{SL-HAC}},description={Single-Linkage Hierarchical Agglomeration Clustering.}}

\glossary{name={\textbf{CPU}},description={Central Processing Unit.}}
\glossary{name={\textbf{GPU}},description={Graphics Processing Unit.}}
\glossary{name={\textbf{GPGPU}},description={General Purpose computing in Graphics Processing Units.}}

\glossary{name={\textbf{API}},description={Application Programming Interface.}}
