\section{EAC Validation}
\label{sec:eac validation}

The present section aims to provide results showing that the proposed methods do not alter the overall quality of the results.
With this in mind, the results of the original version of EAC, implemented in Matlab, are compared with those of the proposed solution.
Several small datasets, chosen from the datasets used by \citet{Lourenco2010} and taken from the UCI Machine Learning repository \cite{Lichman:2013}, were processed by the two versions of EAC.
Furthermore, since the generation of the ensemble is probabilistic and can change the results between runs, the proposed version is processed with the ensembles created by the original version as well.
This guarantees that the combination and recovery phases of EAC, which are deterministic when using SL, are equivalent to the original.
All data in this section refers to processing done in machine Alpha.

Table \ref{tab:validation error acc} presents the difference between the accuracies of the two versions.
Analyzing these results, it is apparent that the difference is minimal.
It should be noted at this point that the original implementation always maps the dissimilarities of the co-association matrix to the range $\left [ 0 , 1 \right ]$.
This forces the co-association matrix to have a floating point data type.
However, since the number of partitions used is usually less than 255, the proposed version uses unsigned integers of 1 byte to reduce the used memory considerably.
The differences in accuracy are thought to come from possible rounding differences and different implementations of core operations between Matlab and Python.

\begin{table}[h]
\centering
\caption{Difference between accuracies from the two implementations of EAC, using the same ensemble. Accuracy was measured using the H-index.}

\begin{tabular}{lll}
\toprule
         &        \multicolumn{2}{c}{Difference between accuracies of implementations} \\
dataset &      True number of clusters & Lifetime criteria \\
\midrule
breast\_cancer &  4.94e-06 &     2.82e-06 \\
ionosphere     &  1.65e-06 &      1.45e-06 \\
iris           &  3.33e-06 &      3.33e-06 \\
isolet         &  1.03e-07 &      4.08e-07 \\
optdigits      &  3.79e-06 &      1.48e-06 \\
pima           &  3.33e-06 &      3.33e-06 \\
pima\_norm     &  4.16e-07 &      4.16e-07 \\
wine\_norm     &  1.12e-07 &      1.91e-06 \\
\bottomrule
\end{tabular}

\label{tab:validation error acc}
\end{table}





% \begin{table}[h]
% \centering
% \caption{Speed-ups obtained in the combination and recovery phases of EAC, using the ensemble generated from the original EAC implementation.}

% \begin{tabular}{lll}
% \toprule
% dataset & Combination & Recovery \\
% \midrule
% breast\_cancer &   7.713564 &        15.22334 \\
% ionosphere    &   9.678288 &        20.12336 \\
% iris          &   14.25549 &         28.4751 \\
% isolet        &   5.500147 &        174.4283 \\
% optdigits     &   9.783604 &        53.21466 \\
% pima          &   85.21744 &        8.406726 \\
% pima\_norm     &   127.2274 &        12.89474 \\
% wine\_norm     &     8.0178 &        12.98206 \\
% \bottomrule
% \end{tabular}

% \label{tab:validation speedup comb rec}
% \end{table}

Table \ref{tab:validation speedup all} presents the speed-up of the proposed version over the original one.
It is clear that speed-up is obtained in all phases of EAC, often by an order of magnitude.
This result, combined with the demonstration that the differences in accuracy are negligible, show that the proposed algorithm performs well in small datasets.

\begin{table}[h]
\centering
\caption{Speed-ups obtained in the different phases of EAC, with independent production of ensembles.}

\begin{tabular}{lccrrrr}
\toprule
dataset &  No. patterns &  No. features & No. classes & Production & Combination & Recovery \\
\midrule
breast\_cancer &           683 &            10 &           2 &      50.4 &    7.5 &        15.8 \\
ionosphere     &           351 &            34 &           2 &      21.9 &   11.3 &        19.9 \\
iris           &           150 &             4 &           3 &      19.8 &   14.5 &        28.5 \\
isolet         &          7797 &           617 &          26 &       7.0 &    6.2 &       206.3 \\
optdigits      &          3823 &            64 &          10 &      17.3 &   10.2 &        53.0 \\
pima           &           768 &             8 &           2 &      50.7 &  141.5 &        13.9 \\
pima\_norm     &           768 &             8 &           2 &      54.3 &  132.9 &        14.4 \\
wine\_norm     &           178 &             4 &           3 &      22.9 &   14.6 &        25.3 \\
\bottomrule
\end{tabular}

\label{tab:validation speedup all}
\end{table}


