\section{Introduction}
\label{intro}
%TODO
% problems under Big Data paradihm
% typical challenges with Big Data
% why EAC?
% combination of both
\subsection{Motivation}
The scope of the thesis is Big Data and Cluster Ensembles.

Success of EAC clustering on hard data sets.

Interesting problems from big data.

Combining the two.

\subsection{Objectives}
The main objectives for this work are:
\begin{itemize}
\item Aplication of Evidence Accumulation Clustering in Big Data (main goal)
\item Exploration of methods that may be included in the EAC paradigm under Big Data constraints (literature review and testing)
\item Study of quantum inspired clustering algorithms and evaluation  of integration in EAC 
\item Study of parallel computation techniques and evaluation of integration in EAC
\item Validation of Big Data EAC on real data (ECG for emotional state discovery and/or heart disease diagnostic)
\end{itemize}

\subsection{Outline}

Explain briefly the work done

The scope of the thesis is Big Data and Cluster Ensembles. A main requirement in this context is to have fast clustering techniques. This may be accomplished in two ways: algorithmically or with parallelization techniques. The former deals with finding faster solutions while the later takes existing solutions and optimizes them with execution speed in mind.

The initial research was under the algorithmic path. More specifically, exploring quantum clustering algorithms. The findings of this exploration were revealed this algorithms to be a poor match for EAC and turned the focus of the research to parallelization techniques. Two main paradigms of parallelization were found appropriate: GPGPU and distributed (among several machines). While the first is a readily available resource in common machines, the second is able to address problems dealing with larger datasets.
